\section{Introduction}
The optimal dynamic delivery of a given fluence map remains a difficult, largely unsolved problem, due to its inherent nonconvexity.
The nonconvexity of the fluence map matching problem leads to a large number of local minima.
Many methods simply ignore this crucial aspect of the problem, and others use a multi-start procedure to find several local minima and return the best one found.
For a thorough introduction to the complexities of dynamic fluence map delivery (which generally arises in the context of volumetric modulated arc therapy, VMAT),
see \cite{balvertcraft} and \cite{unkvmatreview}.

In this technical note, we investigate a regularization procedure which represents leaf trajectories and dose rates as low-order polynomial splines\footnotemark,
such that the optimization procedure naturally focuses on global aspects of the leaf trajectory/dose rate solution.

\footnotetext{A polynomial spline is a function that is formed by a sequence of polynomial segments.}

\subsection{Testing Figures}

\MPKcomment{Koos: I put this section in to make sure that I could get the test figures to compile in the main document.
            I'm not sure where they are actually supposed to go.}

\begin{equation}
y = e^x
\label{eqn:minibixelRefEqnTemplate}
\end{equation}
\todo{Use the correct equation here?}

\begin{figure}[htp]
\centering
    \begin{tikzpicture}
        \pgfplotsset{holdot/.style={color=black,only marks,mark=*,mark size=1.5pt}}
        \pgfplotsset{soldot/.style={color=white,only marks,mark=*,mark size=1.5pt}}
        \begin{axis}[
            xmin=0.5, xmax=10.75,
            ymin=0, ymax=10,
            grid = both,
            grid style = dotted,
            grid style = {line width=.1pt, draw=gray!10},
            major grid style = {line width=.3pt,draw=gray!30},
            axis lines = middle,
            enlargelimits = {abs=0.0},
            axis line style = {-Latex[round]},
            ticklabel style = {font=\footnotesize},
            xticklabel style = {text height=1ex},
            xticklabels = {,$s_L$,$p$,$s_R$, , , $q$, , ,$e_L$,$u$,$e_R$},
            xtick = {1,1.75,2.5,3.25,4,4.75,5.5,6.25,7,7.75,8.5,9.25,10},
            yticklabels = {$0$,$x_p^L$, $x_q^R$, $x_q^L$, $x_u^R$, $T$},
            ytick = {0,2,3.75,6.25,7,8},
            xlabel style = {at={(ticklabel* cs:1)},anchor=north},
            ylabel style = {at={(ticklabel* cs:1)},anchor=east},
            xlabel = {\scriptsize{pos(minibix)}},
            ylabel = {\scriptsize{time(s)}},
        ]

            % Dashed line at height T
            \draw[color=MGHGrey, thick] (0.5,8) -- (10.5,8);

            % Leaf trajectories
            \addplot[fill=black, color=black, only marks, mark=*, mark size = 1.5pt]
                coordinates{(1.75,0.0)(2.5,2.0)(3.25,4.0)(4.0,4.75)(4.75,5.25)(5.5,6.25)(6.25,7.0)(7.0,7.5)(7.75,8.0)};
            \addplot[fill=white, only marks, mark=*, mark size = 1.5pt]
                coordinates{(3.25,0.0)(4.0,1.5)(4.75,3.0)(5.5,3.75)(6.25,4.50)(7.0,5.50)(7.75,6.5)(8.5,7.0)(9.25,8.0)};

            % Different sectors
            \draw[color=MGHGrey, thick, dotted] (1.75,0) -- (1.75,10);
            \draw[color=MGHGrey, thick, dotted] (3.25,0) -- (3.25,10);
            \draw[color=MGHGrey, thick, dotted] (7.75,0) -- (7.75,10);
            \draw[color=MGHGrey, thick, dotted] (9.25,0) -- (9.25,10);

            % Exposure times
            \draw[color=MGHBlue, thick] (2.5,0) -- (2.5,2);
            \draw[color=MGHBlue, thick] (5.5,3.75) -- (5.5,6.25);
            \draw[color=MGHBlue, thick] (8.5,7) -- (8.5,8);

            % Exposure
            \node[color=MGHGrey] at (5.5 ,9.5) {\scriptsize{$g_p = D \times \hdots$}};
            \node[color=MGHGrey] at (1.0 ,8.5) {\scriptsize{$0$}};
            \node[color=MGHGrey] at (2.5 ,8.5) {\scriptsize{$x_p^L$}};
            \node[color=MGHGrey] at (5.5 ,8.5) {\scriptsize{$x_p^L-x_p^R$}};
            \node[color=MGHGrey] at (8.5 ,8.5) {\scriptsize{$T-x_p^R$}};
            \node[color=MGHGrey] at (10.0,8.5) {\scriptsize{$0$}};
        \end{axis}
    \end{tikzpicture}

\caption[Illustration of the total minibixel exposure $g_p$, $p \in \mathcal{P}\backslash\{P\}$]{
    Illustration of the total minibixel exposure $g_p$, $p \in \mathcal{P}\backslash\{P\}$.
    The total minibixel exposure is calculated using four cases, see \addref{miniBixelEqn}.
    Closed and open dots indicate the trajectory of the left and right leaf, respectively.
    }
\label{fig:minibixelExposure}
\end{figure}


% KvA: Example Figure, didn't care too much about efficiency
\begin{figure}[htp]
\centering
    \begin{tikzpicture}[remember picture]
        \pgfplotsset{holdot/.style={color=black,only marks,mark=*,mark size=1.5pt}}
        \pgfplotsset{soldot/.style={color=white,only marks,mark=*,mark size=1.5pt}}
        \begin{axis}[
            xmin=0, xmax=10.50,
            ymin=0, ymax=4.5,
            grid = both,
            grid style = dotted,
            axis x line = bottom,
            axis y line = left,
            enlargelimits = {abs=0.0},
            axis line style = {-Latex[round]},
            yticklabels = {$s$,$x$,$e$},
            ytick = {0.2,2,4},
            xtick = \empty,
            ylabel = {\scriptsize{position (cm)}},
            axis equal image
        ]

            % Vertical lines (top)
            \node[] (traj-1) at (axis cs:1,4.2) {};
            \node[] (traj-2) at (axis cs:1.88,4.2) {};
            \node[] (traj-3) at (axis cs:3.5,4.2) {};
            \node[] (traj-4) at (axis cs:4.17,4.2) {};
            \node[] (traj-5) at (axis cs:5.2,4.2) {};
            \node[] (traj-6) at (axis cs:5.85,4.2) {};
            \node[] (traj-10) at (axis cs:10,4.2) {};

            % Exposure times
            \draw[MGHBlue] (1,2) -- (1.89,2);
            \draw[MGHBlue] (3.5,2) -- (4.17,2.0);
            \draw[MGHBlue] (5.2,2) -- (5.85,2);

            % Leaf trajectories
            \addplot[name path global = leafl, black, smooth, tension=0.75] coordinates{(0,0.2)(1,2)(2.5,3.7)(4.7,1.9)(8.5,4)(10,4)};
            \addplot[name path global = leafr, black, smooth, tension=0.75] coordinates{(0,0.2)(1,0.5)(2.5,2.5)(4.5,1.4)(6.5,2.3)(8,2.1)(10,4)};

        \end{axis}
    \end{tikzpicture}

    \begin{tikzpicture}[remember picture]
        \pgfplotsset{holdot/.style={color=black,only marks,mark=*,mark size=1.5pt}}
        \pgfplotsset{soldot/.style={color=white,only marks,mark=*,mark size=1.5pt}}
        \begin{axis}[
            xmin=0, xmax=10.50,
            ymin=0, ymax=3.0,
            ymajorgrids = true,
            grid style = dotted,
            axis x line = bottom,
            axis y line = left,
            enlargelimits = {abs=0.0},
            axis line style = {-Latex[round]},
            xticklabels = {$T$},
            xtick = {10},
            yticklabels = {$d$},
            ytick = {2.5},
            xlabel style = {at={(ticklabel* cs:0.5,1.5)},anchor=north},
            xlabel = {\scriptsize{time(s)}},
            ylabel = {\scriptsize{dose rate (MU/s)}},
            axis equal image
        ]

            % Vertical lines (bottom)
            \node[] (dose-1) at (axis cs:1,-0.2) {};
            \node[] (dose-2) at (axis cs:1.88,-0.2) {};
            \node[] (dose-3) at (axis cs:3.5,-0.2) {};
            \node[] (dose-4) at (axis cs:4.17,-0.2) {};
            \node[] (dose-5) at (axis cs:5.2,-0.2) {};
            \node[] (dose-6) at (axis cs:5.85,-0.2) {};
            \node[] (dose-10) at (axis cs:10,-0.2) {};

            % Dose rate pattern
            \addplot[name path global = dose, black, smooth, tension=0.75] coordinates{(0,1.25)(2.5,2.4)(6,0.4)(8,1.9)(10,1.7)};\
            \addplot[name path global = xaxis, draw=none, domain=0:11] {0};

            % Integrals
            \addplot [thick, color=blue, fill=MGHGrey, fill opacity=0.2]
                fill between[of=xaxis and dose, soft clip={domain=1:1.88}];
            \addplot [thick, color=blue, fill=MGHGrey, fill opacity=0.2]
                fill between[of=xaxis and dose, soft clip={domain=3.5:4.17}];
            \addplot [thick, color=blue, fill=MGHGrey, fill opacity=0.2]
                fill between[of=dose and xaxis, soft clip={domain=5.2:5.85}];

            % Correction
            \addplot[fill = white]
                fill between[of=xaxis and dose, soft clip={domain=1.88:2}]; % filling
            \addplot[fill = white]
                fill between[of=xaxis and dose, soft clip={domain=4.17:4.5}]; % filling
            \addplot[fill = white]
                fill between[of=xaxis and dose, soft clip={domain=5.85:6}]; % filling
             \draw[black] (0,0) -- (10,0);

        \end{axis}
    \end{tikzpicture}

    \begin{tikzpicture}[remember picture,overlay]
        \draw[dashed,MGHGrey!60] (traj-1) -- (dose-1);
        \draw[dashed,MGHGrey!60] (traj-2) -- (dose-2);
        \draw[dashed,MGHGrey!60] (traj-3) -- (dose-3);
        \draw[dashed,MGHGrey!60] (traj-4) -- (dose-4);
        \draw[dashed,MGHGrey!60] (traj-5) -- (dose-5);
        \draw[dashed,MGHGrey!60] (traj-6) -- (dose-6);
        \draw[dashed,MGHGrey!60] (traj-10) -- (dose-10);
    \end{tikzpicture}

\vspace{-0.7cm}
\caption[Illustration of administered dose]{
    Illustration of administered dose: the dose administered to a position $x$ equals the integral (shaded area) of the dose rate (lower panel) over the moments in time (blue lines) that position is exposed by the leaves (upper panel).
    }
\label{fig:administeredDose}
\end{figure}


In this section we have a figure about minibixel exposure \ref{fig:minibixelExposure} and about administered dose \ref{fig:administeredDose}.

\subsection{Fluence Mapping as Continuous Optimization}

Our goal is to compute the optimal leaf trajectories $x_L(t)$ and $x_U(t)$, as well as the optimal dose rate $r(t)$ of the radiation source.
For this initial study we will consider only a single leaf pair, although our algorithm is structured such that it could be applied to an arbitrary number of leaf pairs.

\MPKcomment{What does the following mean?  {\em REMARK: assuming independence of rows, so no tongue-and-groove effect.}}

We will assume that the desired fluence dose at each point $q(x)$ on the subject is given.
The optimization is given below, where $q'(x)$ is the fluence dose that is delivered by the
dose rate and leaf trajectories and $X$ is the position domain of the fluence map.

\begin{equation}
\underset{r(t), \, x_L(t), \, x_U(t)}{\operatorname{argmin}}
\int_X \bigg(q'(x)) - q(x)\bigg)^2 dx
\label{eqn:fluenceMapOptimization}
\end{equation}

The fluence dose at each location can be computed as

\begin{equation}
q'(x) = \int_{T(x)} r(t) dt
\label{eqn:deliveredFluenceDose}
\end{equation}

where $T(x)$ is the set of times when the position $x$ is not blocked by the leaves:
$T(x)$ is the set of all times $t$ such that $ x_L(t) < x < x_U(t)$.


% We will also assume that the leaves start next to each other:
% $x_L(0) = x_U(0)$ and
% $x_L(T) = x_U(T)$ where
% $t \in [0, T]$.

\MPKcomment{Should we assume that the leaf boundary conditions are known?
            Do the leaves need to start and end together?}
