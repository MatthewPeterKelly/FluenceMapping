\section{Introduction}
The optimal dynamic delivery of a given fluence map remains a difficult, largely unsolved problem, due to its inherent nonconvexity.
The nonconvexity of the fluence map matching problem leads to a large number of local minima. 
Many methods simply ignore this crucial aspect of the problem, and others use a multi-start procedure to find several local minima and return the best one found. 
For a thorough introduction to the complexities of dynamic fluence map delivery (which generally arises in the context of volumetric modulated arc therapy, VMAT),
see \cite{balvertcraft} and \cite{unkvmatreview}.

In this technical note, we investigate a regularization procedure which represents leaf trajectories and dose rates as low-order polynomials,
such that the optimization procedure naturally focuses on global aspects of the leaf trajectory/dose rate solution.

\subsection{Fluence Mapping as Continuous Optimization}

Our goal is to compute the optimal leaf trajectories $x_L(t)$ and $x_U(t)$, as well as the optimal dose rate $r(t)$ of the radiation source.
For this initial study we will consider only a single leaf pair, although our algorithm is structured such that it could be applied to an arbitrary number of leaf pairs.
% REMARK: assuming independence of rows, so no tongue-and-groove effect
We will assume that the desired fluence dose at each point $q(x)$ on the subject is given.
The optimization is given below, where $q'(x)$ is the fluence dose that is delivered by the
dose rate and leaf trajectories and $X$ is the position domain of the fluence map.

\begin{equation}
\underset{r(t), \, x_L(t), \, x_U(t)}{\operatorname{argmin}}
\int_X \bigg(q'(x)) - q(x)\bigg)^2 dx
\label{eqn:fluenceMapOptimization}
\end{equation}

The fluence dose at each location can be computed as

\begin{equation}
q'(x) = \int_{T(x)} r(t) dt
\label{eqn:deliveredFluenceDose}
\end{equation}

where $T(x)$ is the set of times when the position $x$ is not blocked by the leaves:
$T(x)$ is the set of all times $t$ such that $ x_L(t) < x < x_U(t)$.
