\section{Introduction}
The optimal dynamic delivery of a given fluence map remains a difficult, largely unsolved problem, due to its inherent nonconvexity.
The nonconvexity of the fluence map matching problem leads to a large number of local minima. Many methods simply ignore this crucial aspect
of the problem, and others use a multi-start procedure to find several local minima and return the best one found. For a thorough introduction
to the complexities of dynamic fluence map delivery (which generally arises in the context of volumetric modulated arc therapy, VMAT),
see \cite{balvertcraft} and \cite{unkvmatreview}.

In this technical note, we investigate a regularization procedure which represents leaf trajectories and dose rates as low-order polynomials,
such that the optimization procedure naturally focuses on global aspects of the leaf trajectory/dose rate solution.


\subsection{Fluence Mapping as Continous Optimization}

Our goal is to compute the optimal leaf trajectories $x_L(t)$ and $x_U(t)$, as well as the optimal dose rate $r(t)$ of the radiation source.
For this initial study we will consider only a single row of leaves,
although our algorithm is structured such that it could be applied to an arbitrary number of leaf rows.
We will assume that the desired fluence dose at each point $q(x)$ on the subject is given.
The optimization problem can thus be stated:

\begin{equation}
x(t)
\end{equation}
