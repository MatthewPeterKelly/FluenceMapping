\section{Introduction}
The optimal dynamic delivery of a given fluence map remains a difficult,
largely unsolved problem, due to its inherent nonconvexity.
The nonconvexity of the fluence map matching problem leads to a large number of local minima.
Many methods simply ignore this crucial aspect of the problem,
and others use a multi-start procedure to find several local minima and return the best one found.
For a thorough introduction to the complexities of dynamic fluence map delivery
(which generally arises in the context of volumetric modulated arc therapy, VMAT),
see \cite{balvertcraft} and \cite{unkvmatreview}.

In this technical note, we investigate a regularization procedure which represents
leaf trajectories and dose rates as linear splines (piece-wise linear functions),
such that the optimization procedure naturally focuses on global aspects of the leaf trajectory/dose rate solution.


\subsection{Background}

\todo{ David  --  Would you be able to add any background that is appropriate for this paper,
at least from a fluence mapping perspective?  Let me know if I should add a short section about
trajectory optimization. Also, should background be its own section, or is a sub-section in the
Introduction acceptable?}

\subsection{Fluence Mapping as Continuous Optimization}
\label{sec:FluenceMappingAsContinuousOptimization}

Our goal is to compute the optimal leaf trajectories $z_L(t)$ and $z_U(t)$,
as well as the optimal dose rate $r(t)$ of the radiation source.
For this initial study we will consider only a single leaf pair,
although our algorithm is structured such that it could be applied to an arbitrary number of leaf pairs.
We will assume that adjacent leaves are independent of each other,
neglecting the small coupling terms created by the tongue-and-groove mechanism on the real machine.

We will assume that the desired fluence dose at each point $q_D(x)$ on the subject is given,
and that we would like to compute the dose-rate $r(t)$ and leaf trajectories $z_L(t)$ and $z_U(t)$ that
minimize the error between the fluence that is predicted to reach the target $q(x)$ and the desired
fluence $q_D(x)$.

\begin{equation}
\underset{r(t), \, x_L(t), \, x_U(t)}{\operatorname{argmin}}
\int_X \bigg(q(x)) - q_D(x)\bigg)^2 dx
\label{eqn:fluenceMapOptimization}
\end{equation}

\vspace{6pt}

The fluence dose at each location is the time-integral of the dose rate trajectory for the times
when the radiation is not being blocked by the upper or lower leaves.
This time domain $T(x)$ is the set of times when the position $x$ is not blocked by the leaves:
$T(x)$ is the set of all times $t$ such that $ x_L(t) < x < x_U(t)$.

\begin{equation}
q(x) = \int_{T(x)} r(t) dt
\label{eqn:deliveredFluenceDose}
\end{equation}

For this work we will leave the initial and final leaf positions as decision variables
in the optimization, but it is a trivial extension to enforce arbitrary initial and final
leaf positions with a simple modification to the position limits at these points in the
trajectory. This might be useful when combining a sequence of fluence maps as shown in
\addref{Multi-VMAT paper}.

We will also assume that the total duration of the trajectories is given,
but this could easily be added as a decision variable at a later time.

% $x_L(0) = x_U(0)$ and
% $x_L(T) = x_U(T)$ where
% $t \in [0, T]$.


\subsection{Implementation as a Nested Optimization}

The optimization formulation presented in Section \S \ref{sec:FluenceMappingAsContinuousOptimization}
is a good mathematical description of our goal, but it is not obvious how to solve it.
In this paper we use a nested optimization approach,
solving the dose-rate trajectory $r(t)$ in an outer optimization,
and then solving the optimal leaf trajectories $z_L(t)$ and $z_U(t)$ as an inner optimization,
given a candidate dose-rate trajectory.

Partitioning the problem in this fashion makes physical sense: given a fixed dose-rate trajectory,
the leaf trajectories are completely independent.
This decoupling allows us to focus on designing the inner and outer optimizations differently,
each specialized for the details of that optimization.

\todo{David - do we need an equation here?
    If so, I'm not sure of the best way to write the nested optimization.}
