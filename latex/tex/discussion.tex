\section{Discussion}

\subsection{Why first-order splines to represent trajectories?}
\label{sec:WhyUseLinearSplines}

In this section we informally discuss some of the design choices that were made when creating this
trajectory-based method for solving leaf and dose-rate trajectories for fluence mapping.

\todo{This section should be edited to reduce length and needs some detailed citations.}

There are many ways to represent trajectories, and nearly all of them have associated transcription (optimization) methods.
In general there is one major trade-off:
use a larger number of low-order segments or
use a smaller number of high-order segments.

Selecting the correct method order typically comes down to problem specifics.
High-order methods are best when the system model is good and high accuracy is important.
The accuracy of these methods relies on the underlying problem being smooth over each
mesh interval, with few places where path constraints serve to shape the trajectory.
Low-order methods are best when the trajectory shape is dominated by path constraints and
high accuracy is not critical.
\todo{Cite my tutorial paper? Check if I discuss this in detail.
Also see if it is discussed in the reviews by Rao or Betts.}

Although our specific problem does not have path constraints, it does have a highly nonlinear objective function.
The smooth leaf-blocking model causes a strong nonlinearity in the objective function,
which would make careful mesh refinement necessary if high-order methods were used.
\todo{Cite GPOPS-II paper? Or perhaps Rao or Betts tutorial paper.}
We can avoid the need for careful mesh refinement by simply using a low-order model,
which also makes it easier to precisely handle the velocity and position limits on the leaf trajectories.

We did a few brief checks, comparing a few different schemes using cubic splines and compared them
to the linear spline presented in this paper. Although both methods worked, the linear spline methods
were much faster and were able to provide an good fit to the desired fluence profile.
One possible reason for the speed increase is that a set of three linear segments have less
coupling terms than a cubic spline fitting the same trajectory domain. The resulting sparsity in the
jacobian (derivative of the objective function) helps optimization. \cite{Betts}

The major drawback of a linear spline for leaf trajectories is that the resulting trajectory tends
to be less smooth, with a step change in velocity between each segment. This effect can be minimized
by using a large number of linear segments. In some cases this can lead to a slight numerical
instability, which is easily addressed by adding a small regularization term to the optimization,
minimizing the integral of the velocity-squared along the trajectory.


\subsection{Inner Optimization:  Leaf Trajectories}

\todo{Discuss figures in results section.}


\subsection{Outer Optimization:  Dose-Rate Trajectories}

\todo{Try doing a better job of performing two optimizations, and varying the leaf smoothing between them.
      See if this improves results.}

The outer optimization with CMAES works, but it still has the problem with local minima:
running the optimization N times produces several different solutions.
These solutions all have similar objective function values, but are created by different
dose-rate trajectories.

It seems to me that the dose-rate optimization has a large number of similar-valued minima,
since the leaf-trajectory optimization can find good solutions for most dose-rate trajectories.
The residual fitting error seems to be more related to the choice of discretization -- number of
grid segments, rather than convergence failure.

I see two ways to go forward from here. First would be to try this optimization with a set of
leaf trajectories, to see if that helps force a better global solution, since the problem will
be more constrained. Second would be to play around more with global optimization and smoothing
terms.

If I make the smoothing term on the dose-rate trajectory large, then I get repeatable solutions,
but they tend to be nearly-constant dose rate, and often relatively high.

Another option would be to modify the regularization term to include a  penalty on large
dose rates, which is perhaps desirable in the real system.

\todo{final results pending}
