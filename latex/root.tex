\documentclass[12pt]{article}
\usepackage{amsmath,amsfonts,amssymb}   %% AMS mathematics macros
\usepackage{graphicx}
\usepackage{fancyvrb}
\usepackage{color}
\usepackage{url}
\usepackage{hyperref}
\usepackage{authblk}
\usepackage{geometry}
\linespread{1.4}
 \geometry{
 a4paper,
 total={174mm,257mm},
 left=20mm,
 top=20mm,
 }
 \renewcommand{\floatpagefraction}{.8}
 \renewcommand\Affilfont{\fontsize{9}{10.8}\itshape}
 
 \title{Dynamic leaf sequencing via polynomial representations of leaf trajectories and dose rate}

 \author{Matthew Kelly, Koos van Amerongen, David Craft}
\affil[]{Department of Radiation Oncology, Massachusetts General Hospital, Harvard Medical School}
%% \date{February 13, 2016}

\begin{document}

\maketitle
\thispagestyle{empty}

%\begin{figure}[!htbp]
%\centering
%%trim option's parameter order: left bottom right top  
%\includegraphics[trim=0 50 0 0,clip,width=17cm]{clonalTree}
%\caption{Clonal expansion of a cancer cell population. The population starts exponentially dividing with fitness $f_0$ at time 0.}
%\label{clonalTree}
%\end{figure}


\begin{abstract}
  We consider the problem of matching a given fluence map $f$ to a close approximation in limited time $T$ by the use
  of a multi-leaf collimator, i.e. the setting of IMRT and VMAT in radiation therapy. We use low order polynomial 
  representations for leaf velocities and the dose rate as functions of time. The optimization then searches over the coefficients
  that define those polynomials, resulting in an optimization problem in a smaller dimension and one that is inherently smoother,
  thus focusing on global aspects of the solution and avoiding the noisy local minima that occur when solving the problem in the original
  space of leaf positions and dose rates vs time.
\end{abstract}

\section{Introduction}
The optimal dynamic delivery of a given fluence map remains a difficult, largely unsolved problem, due to its inherent nonconvexity.
The nonconvexity of the fluence map matching problem leads to a large number of local minima. Many methods simply ignore this crucial aspect
of the problem, and others use a multi-start procedure to find several local minima and return the best one found. For a thorough introduction
to the complexities of dynamic fluence map delivery (which generally arises in the context of volumetric modulated arc therapy, VMAT),
see \cite{balvertcraft} and \cite{unkvmatreview}.

In this technical note, we investigate a regularization procedure which represents leaf trajectories and dose rates as low-order polynomials,
such that the optimization procedure naturally focuses on global aspects of the leaf trajectory/dose rate solution.


\section{Methods}

[MATT this is copy and pasted from the original ``chebyopt.pdf'' that I sent you. Probably all will need to be revised, but leaving it in case some is salvagable/usable.]

We consider a single row version of the problem. Extension to the full fluence map should be possible, but it makes sense
to first develop the method and see how it works for a single row. Let $f(x)$ be the desired fluence map to deliver, where
$x$ is the position across the fluence row, $x \in [-1,1]$. We choose this as the domain because that is the
domain that the Chebyshev polynomials are written for. Across that domain the Chebyshev polynomials range from $-1$ to $1$.
We assume we have $T=1$ time unit to reproduce the fluence map $f$ by sliding leaves across the field from left to right.
Let $v_L(t)$ be the velocity of the left leaf at time $t$ and $v_R(t)$ the right leaf velocity. We assume we fix the starting positions
of both leaves. We may end up assuming we fix the ending positions too. This is not ideal, in fact we would like to optimize over
starting and ending leaf positions as well, but to make things more concrete, for now starting positions are assumed fixed and given.

Let $x_L(t)$ be position of the left leaf at time $t$. We have
\begin{equation}
  x_L(t) = x_L(0) + \int_{\tau=0}^t v_L(\tau) d\tau. 
\end{equation}
\noindent (ditto for right leaf). Let $T_L(x)$ be the inverse of this function
(if we assume/enforce that velocity is always $>0$ then this inverse exists). Thus
$T_L(x)$ is the time that the left leaf crosses the position $x$. The fluence delivered to position
$x$, which we call $g(x)$ is the integration of the dose rate over the interval for which the position $x$ is not covered by either leaves, see Figure \ref{sketch}.

%\begin{figure}[!htbp]
%\centering
%%trim option's parameter order: left bottom right top
%\includegraphics[trim=0 0 0 0,clip,width=12cm]{sketchCheby.pdf}
%\caption{Sketch of two leaf trajectories and the integration range (dotted line) for computing $g(x)$.}
%\label{sketch}
%\end{figure}

\begin{equation}
  g(x) = \int_{t = T_R(x)}^{T_L(x)} d(t) dt
\end{equation}
\noindent where d(t) is the dose rate at time $t$.

The optimization problem is to find the velocities and the dose rate over time to get $g$ as close to $f$ as possible:

\begin{equation}
\mathrm{min} \int_{x=-1}^1 \left ( g(x) - f(x) \right )^2 dx
\label{opt}
\end{equation}

ENTER THE CHEBYSHEVS: We represent the velocities $v$ as weighted sums of Chebyshev polynomials of the first kind.
Here are the first four, for ease of reference:
$$
T_0(t) = 1
$$
$$
T_1(t) = t
$$
$$
T_2(t) = 2t^2 -1
$$
$$
T_3(t) = 4t^3 - 3t
$$

%$$[ 1, x, 2*x^2 - 1, 4*x^3 - 3*x, 8*x^4 - 8*x^2 + 1]

So, for example, the left leaf velocity is given by
$$
v_L(t) = \sum_{i=0}^3 c^L_i T_i(t).
$$
\noindent and we can do the same for the right leaf velocity and the dose rate as a function of time.
The optimization problem them becomes a search for the coefficients $c^L_i$, $c^R_i$, and $c^d_i$ that minimize the objective in Equation \ref{opt}.

\section{Results}

TODO:  results

\section{Discussion and conclusions}

\bibliographystyle{unsrt}
\bibliography{all}

\end{document}


