\documentclass{standalone}
\usepackage{graphicx,amsmath}
\usepackage{amssymb,amsthm}
\usepackage{varwidth}
\usepackage{moresize}
\usepackage[skins]{tcolorbox}
\usepackage{tikz}
\usetikzlibrary{external}
\usetikzlibrary{calc}
\usepackage{pgfplots}
\pgfplotsset{compat=newest} % Allows to place the legend below plot

\begin{document}
	\begin{tikzpicture}[x=1cm, y=1cm]
    \begin{axis}[
          clip=false,
          clip mode=individual,
          width=\linewidth, % Scale the plot to \linewidth
          height=150pt,
          xmin=0, xmax=7,
          ymin=0, ymax=12,
          enlarge x limits=false,
          enlarge y limits=false,
          axis lines=center,
          axis line style={-},
          grid=both,
          grid style={line width=.2pt, draw=gray!30},
          major grid style={line width=.3pt,draw=gray!60},
          minor x tick num=1,
          xlabel style = {anchor=west,
            at={(ticklabel* cs:1.01,0)},
            text width=width("Max delivery"),
          },
          ylabel style={at={(ticklabel* cs:1.01)},
            anchor=south
          },
          xlabel={Max delivery \newline time (s)},
          ylabel=ssdif ($\times 10^4$),
          x tick label style={anchor=north},
          y tick label style={anchor=east},
          xtick distance=1,
          scaled ticks=false,
          max space between ticks=40pt, % controls % y ticks
          try min ticks = 5
        ]

    % Plot the dots and connect
    \addplot[black, mark=*] coordinates {(1,12.000) (2,2.500) (3,1.200) (4,0.450) (5,0.200) (6,0.060) (7,0)};

    % Keep working in same enviroment, convert to cm
    \def\Ysc{12/150/0.0353}  % maxY to cm (via pt)
    \def\eps{0.03}

    % Original map  
    \node[anchor=west] at (7.0,5.6*\Ysc) {\footnotesize{Original map:}}; % Text
    \path[fill stretch image=aFluenceMap] (7.2,2.8*\Ysc) rectangle (8.2-2*\eps,5.3*\Ysc); % Figure
    %\node[inner sep=0pt, anchor=west] (org) at (6,6000) {\pgftext{\includegraphics[width=1cm]{aFluenceMap.pdf}}};   % FIX WIDTH AND HEIGHT ISSUE

    % Draw the fluence maps
    \node[rotate=90, text width = width("Fluence"), text centered] at (0,-2.05*\Ysc) {\footnotesize{Fluence map}};
    \path[fill stretch image=aFluenceMap] (0.5+\eps,-3.3*\Ysc) rectangle (1.5-\eps,-0.8*\Ysc);
    \path[fill stretch image=aFluenceMap] (1.5+\eps,-3.3*\Ysc) rectangle (2.5-\eps,-0.8*\Ysc);
    \path[fill stretch image=aFluenceMap] (2.5+\eps,-3.3*\Ysc) rectangle (3.5-\eps,-0.8*\Ysc);
    \path[fill stretch image=aFluenceMap] (3.5+\eps,-3.3*\Ysc) rectangle (4.5-\eps,-0.8*\Ysc);
    \path[fill stretch image=aFluenceMap] (4.5+\eps,-3.3*\Ysc) rectangle (5.5-\eps,-0.8*\Ysc);
    \path[fill stretch image=aFluenceMap] (5.5+\eps,-3.3*\Ysc) rectangle (6.5-\eps,-0.8*\Ysc);
    \path[fill stretch image=aFluenceMap] (6.5+\eps,-3.3*\Ysc) rectangle (7.5-\eps,-0.8*\Ysc);

    % Draw the colorbar
    \path[fill stretch image=aFluenceMap] (7.5+2*\eps,-3.3*\Ysc) rectangle (8.5-\eps,-0.8*\Ysc);

    % Draw the dose rate axis
    \node[rotate=90, text width = width("Dose rate"), text centered] at (0,-4.5*\Ysc) {\footnotesize{Dose rate (MU/s)}};
    \node[anchor = north, text centered] at (4,-5.9*\Ysc) {\footnotesize{Time (s)}};
    \draw[-](0.5+\eps,-5.5*\Ysc)--(1.5-\eps,-5.5*\Ysc);
    \draw[-](1.5+\eps,-5.5*\Ysc)--(2.5-\eps,-5.5*\Ysc);
    \draw[-](2.5+\eps,-5.5*\Ysc)--(3.5-\eps,-5.5*\Ysc);
    \draw[-](3.5+\eps,-5.5*\Ysc)--(4.5-\eps,-5.5*\Ysc);
    \draw[-](4.5+\eps,-5.5*\Ysc)--(5.5-\eps,-5.5*\Ysc);
    \draw[-](5.5+\eps,-5.5*\Ysc)--(6.5-\eps,-5.5*\Ysc);
    \draw[-](6.5+\eps,-5.5*\Ysc)--(7.5-\eps,-5.5*\Ysc);

    \draw[-](0.5+\eps,-5.5*\Ysc)--(0.5+\eps,-3.5*\Ysc);
    \draw[-](1.5+\eps,-5.5*\Ysc)--(1.5+\eps,-3.5*\Ysc);
    \draw[-](2.5+\eps,-5.5*\Ysc)--(2.5+\eps,-3.5*\Ysc);
    \draw[-](3.5+\eps,-5.5*\Ysc)--(3.5+\eps,-3.5*\Ysc);
    \draw[-](4.5+\eps,-5.5*\Ysc)--(4.5+\eps,-3.5*\Ysc);
    \draw[-](5.5+\eps,-5.5*\Ysc)--(5.5+\eps,-3.5*\Ysc);
    \draw[-](6.5+\eps,-5.5*\Ysc)--(6.5+\eps,-3.5*\Ysc);
    
    % Draw the dose rate ticks 
    % y-ticks
    \node[anchor = east] at (0.5+\eps,-5.5*\Ysc) {\footnotesize{0}};
    \node[anchor = east] at (0.5+\eps,-3.5*\Ysc) {\footnotesize{10}};

    % x-ticks
    \node[anchor = north] at (0.5+3*\eps,-5.5*\Ysc) {\footnotesize{0}};
    \node[anchor = north] at (1.5-3*\eps,-5.5*\Ysc) {\footnotesize{1}};
    \node[anchor = north] at (1.5+3*\eps,-5.5*\Ysc) {\footnotesize{0}};
    \node[anchor = north] at (2.5-3*\eps,-5.5*\Ysc) {\footnotesize{2}};
    
    % Formula for each coordinate: Y-value: -5.5+2*(dr/maxdr)  % remember, 0 = -5, 10 = -3
    %                              X-value: (T-1)+0.5 + (1-2*\eps)*(maxt - t) % remember, 0.5 = 1, 1.5 = maxVal, +1 per T
    % etc.
    
    % Draw the dose rate patterns
    \addplot[black, mark=*, mark size=1pt] coordinates {(0.5+\eps,-3.5*\Ysc) (1,-3.5*\Ysc) (1.5-\eps,-3.9*\Ysc)};
    \addplot[black, mark=*, mark size=1pt] coordinates {(1.5+\eps,-3.5*\Ysc) (1.75,-4.2*\Ysc) (2,-3.6*\Ysc) (2.25,-5.3*\Ysc) (2.5-\eps,-3.9*\Ysc)};
    % etc.

    % These are the areas we can play inside
%    \draw[->](0,-0.8*\Ysc)--(7,-0.8*\Ysc);
%    \draw[->](0,-3.3*\Ysc)--(7,-2.8*\Ysc);
%    \draw[->](0,-3.5*\Ysc)--(7,-3.0*\Ysc);
%    \draw[->](0,-5.5*\Ysc)--(7,-5.0*\Ysc);

    \end{axis}
	\end{tikzpicture}
\end{document}

% TODO: convert this all into Matlab-code 