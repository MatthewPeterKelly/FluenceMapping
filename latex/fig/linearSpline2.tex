% KvA: Example Figure, didn't care too much about efficiency

\begin{figure}[htp]
\centering
\begin{tikzpicture}
    \begin{axis}[
        xmin=0, xmax=6.0,
        ymin=0, ymax=2.4,
        axis x line = bottom,
        axis y line = left,
        enlargelimits = {abs=0.2},
        axis line style = {-Latex[round]},
        ytick = \empty,
        xtick = \empty,
        ylabel = {value},
        xlabel = {time},
        axis equal image
        ]
        
    \addplot[   thick,
        color=black,
        solid,
        mark=o,
        mark size=3,
        mark options={solid},
        visualization depends on=\thisrow{alignment} \as \alignment,
        nodes near coords, % Place nodes near each coordinate
        point meta=explicit symbolic, % The meta data used in the nodes is not explicitly provided and not numeric
        every node near coord/.style={yshift=\alignment, anchor=north} % Align each coordinate at the anchor 40 degrees clockwise from the right edge
        ] table [% Provide data as a table
            meta index=2 % the meta data is found in the third column
        ] {
        x       y       label       alignment
        0.5     0.45    $(t_0,x_0)$       -1
        1.5     1.25    $(t_1,x_1)$       -9
        2.5     1.75    $(t_2,x_2)$       -8
        3.5     1.0     $(t_3,x_3)$       -3
        4.5     0.8     $(t_4,x_4)$       -3
        5.5     1.4     $(t_5,x_5)$       -6
        };

        \node at (2.5,1.75) (knot) {};
        \node[anchor=west] at (3.5,2.25) (description) {knot point};
        \draw[color=gray, thick] (description) edge[out=180,in=45,-latex] (knot);

    \end{axis}
\end{tikzpicture}
    
    
\caption[Linear Spline]{ 
    Dose rate and leaf position trajectories are represented using linear splines.  
    A linear spline is fully defined by its values at the knot points. 
}
\label{fig:linearSpline}
\end{figure}
