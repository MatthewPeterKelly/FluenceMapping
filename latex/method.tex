\section{Method}


\subsection{Trajectory Representation}

The final goal of our optimization is to compute three trajectories:
radiation dose-rate $r(t)$, lower leaf position $z_L(t)$, and upper leaf position $z_U(t)$.
Each of these trajectories is represented by a piece-wise linear spline,
with shared knot times $t_k \in \{t_0, t_1, \dots, t_N\}$, where $N$ is the number of knot points.
The value of the dose rate and leaf positions at knot time $t_k$ are given by
$r_k$, $x_{L,k}$, and $y_{U,k}$ respectively.

\todo{add figure for linear spline?}

\subsection{Trajectory Limits}

There are several constant limits that must be enforced along the trajectory
to ensure that we can run the solution trajectory on the VMAT machine.
\MPKcomment{Is this correct usage of VMAT machine?}
These limits are detailed below, where $\dot{z}(t) = \tfrac{d}{dt}z(t)$ gives the leaf velocity.
\begin{align}
  \text{dose-rate: }& \quad 0 \leq r(t) \leq r_\text{max} \\
  \text{leaf position: }& \quad 0 \leq z_L(t) \leq z_U(t) \leq x_\text{max} \\
  \text{lower leaf velocity: }& \quad v_\text{min} \leq \dot{z}_L(t) \leq v_\text{max} \\
  \text{upper leaf velocity: }& \quad v_\text{min} \leq \dot{z}_U(t) \leq v_\text{max}
\end{align}

\subsection{Avoiding Local Minima}

One of the key issues with fluence mapping is that the
optimization tends to get stuck in local minima,
since there are a large number of leaf trajectories that deliver the same fluence to the target.
\addref{local minima in fluence mapping}
Of these many solutions, the most desirable solutions are those which have smooth leaf trajectories,
because they reduce wear on the machine and less likely to exploit inaccuracies in the model.
\MPKcomment{Need some reference here, or perhaps a better explanation of why smooth is good.}
One way to create a smooth trajectory is to minimize the integral of the derivative-squared,
a regularization trick that is commonly used in trajectory optimization of dynamical systems.
\addref{torque-squared optimization}
For our problem, this regularization term can be written:
\begin{equation}
  J_\text{vel}\big(\dot{z}_L(t), \dot{z}_U(t)\big)
    = \int_0^T \! \big( \dot{z}_L^2(t) + \dot{z}_U^2(t) \big) \,dt
\end{equation}

\subsection{Computing Delivered Fluence}

The fluence delivered at each position on the target is computed by integrating the
dose-rate function over the periods of time when the leaves are not blocking the target:

\begin{equation}
  f_D(x) = \int_{\mathcal{T}(x)} \! r(t) \,dt
  \quad \quad
  \mathcal{T}(x) = \forall t
  \quad
  \text{S.T.}
  \quad
  z_L(t) \leq x \leq z_U(t)
  \label{eqn:fluenceMapIntegral}
\end{equation}
\MPKcomment{Is there a better way to write the definition of $\mathcal{T}$?}

It is tricky to compute this integral directly, as it would require computing
the inverse of the leaf trajectories.
In addition, it creates a discontinuity in the gradient
(\textit{e.g.} $\tfrac{\partial f_D}{\partial z_L})$)
which can cause issues when putting this integral inside of an objective function.

Our solution is to rewrite the integral using a blocking function $k(t)$,
which has a value of one when the leaves are passing radiation and
zero when the leaves are blocking radiation.
This allows us to rewrite (\ref{eqn:fluenceMapIntegral}) to use simple bounds:

\begin{equation}
  f_D(x) = \int_0^T \! k(t, x) \cdot r(t) \, dt
\end{equation}

A simple way to define $k(t)$ would be to set it to one if $z_L(t) \leq x \leq z_U(t)$
is true, and zero otherwise.
This implementation would have a discontinuous gradient, which would cause problems in the optimization.
Instead, we use a polynomial smoothing to approximate the step function $s(x,\alpha)$,
where $\alpha$ is a smoothing parameter.
Note that the function $s(x,\alpha)$ has continuous value, slope, and curvature.

\begin{equation}
  k(t, x) = s\big(x - z_L(t), \alpha\big) \; \cdot \; s\big(z_U(t) - z, \alpha\big)
\end{equation}

\begin{equation}
  s(x, \alpha) =
    \begin{cases}
      0, & \text{for } x \leq -\alpha \\
      6\tau^5-15\tau^4+10\tau^3, & \text{for }  -\alpha < x < \alpha\\
      1, & \text{for } x \geq \alpha
    \end{cases}
\end{equation}

\begin{equation}
  \tau(x) = \frac{x + \alpha}{2\alpha}
\end{equation}
