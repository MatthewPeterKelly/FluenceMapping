\section{Method}

\subsection{Trajectory Representation}

The general algorithm presented in this paper is agnostic to the choice of trajectory representation.
There is a large literature in trajectory optimization studying trajectory representation. \addref{trajectory representation}
Here we choose to use shape-preserving cubic splines: the \texttt{PCHIP} spline type in Matlab.
This choice of trajectory has a few useful properties: it is continuous in leaf position and velocity
and it makes it simple to represent the trajectory using only the position at a few knot points.
A piecewise linear spline would also be a reasonable choice.

\todo{why cubic $>$ linear?  Would linear be faster?}
\KvAcomment{If linear spline is indeed faster, I'd go for that. With low-order polynomials that shouldn't harm too much.}

\subsection{Nested Optimization}

This optimization problem has a natural structure.
Rather than solve both the dose rate and leaf trajectories simultaneously,
we instead optimize the dose rate as an outer optimization, and the leaf trajectories as an inner optimization.
Although we only solve a single leaf trajectory in this paper, this decoupling allows these methods to easily extend to situations with multiple leaves.

\todo{Decoupled optimization equations}

\subsection{Computing Delivered Fluence}

One of the core numerical methods in this paper is the technique for computing an estimate of the fluence
that will be delivered (\ref{eqn:deliveredFluenceDose}), given a dose rate and leaf trajectories.

The first step is to compute the time spans where the radiation is able to pass between the leaves: $T(x)$.
\todo{explain the algorithm for this}.
Once this is done, then we can replace the original integral (composite domain) with a set of definite integrals,
each of which has a simple domain:

\begin{equation}
q'(x) = \int_{T(x)} r(t) dt = \int_{t_0}^ {t_1} + \dots + \int_{t_{N-1}}^ {t_N}
\label{eqn:deliveredFluenceDoseSum}
\end{equation}

Each of these integrals can then be computed using standard quadrature techniques.
Here we choose to use Simpson quadrature.
