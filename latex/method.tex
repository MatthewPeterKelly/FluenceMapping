\section{Method}

\subsection{Trajectory Representation}

The general algorithm presented in this paper is agnostic to the choice of trajectory represention.
There is a large literature in trajectory optimization studying trajectory representation. \addref{trajectory representation}
Here we choose to use shape-preserving cubic splines: the \texttt{PCHIP} spline type in Matlab.
This choice of trajectory has a few useful properties: it is continuous in leaf position and velocity
and it makes it simple to represent the trajectory using only the position at a few knot points.
A piecewise linear spline would also be a reasonable choice.

\todo{why cubic $>$ linear?  Would linear be faster?}

\subsection{Nested Optimization}

This optimization problem has a natural structure. Rather than solve both the dose rate and leaf trajectories simultaneously,
we instead optimize the dose rate as an outer optimization, and the leaf trajectories as an inner optimization.
Although we only solve a single leaf trajectory in this paper, this decoupling allows these methods to easily extend to situations with multiple leaves.

\todo{Decoupled optimization equations}

\subsection{Computing Delivered Fluence}

One of the core numerical methods in this paper is the technique for computing an estimate of the fluence
that will be delivered (\ref{eqn:deliveredFluenceDose}), given a dose rate and leaf trajectories.

The first step is to compute the time spans where the radiation is able to pass between the leaves: $T(x)$.
\todo{explain the algorithm for this}.
Once this is done, then we can replace the original integral (composite domain) with a set of definite integrals,
each of which has a simple domain:

\begin{equation}
q'(x) = \int_{T(x)} r(t) dt = \int_{t_0}^ {t_1} + \dots + \int_{t_{N-1}}^ {t_N}
\label{eqn:deliveredFluenceDoseSum}
\end{equation}

Each of these integrals can then be computed using standard quadrature techniques.
Here we choose to use Simpson quardature.

\subsection{template:}

We consider a single row version of the problem. Extension to the full fluence map should be possible, but it makes sense
to first develop the method and see how it works for a single row. Let $f(x)$ be the desired fluence map to deliver, where
$x$ is the position across the fluence row, $x \in [-1,1]$. We choose this as the domain because that is the
domain that the Chebyshev polynomials are written for. Across that domain the Chebyshev polynomials range from $-1$ to $1$.
We assume we have $T=1$ time unit to reproduce the fluence map $f$ by sliding leaves across the field from left to right.
Let $v_L(t)$ be the velocity of the left leaf at time $t$ and $v_R(t)$ the right leaf velocity. We assume we fix the starting positions
of both leaves. We may end up assuming we fix the ending positions too. This is not ideal, in fact we would like to optimize over
starting and ending leaf positions as well, but to make things more concrete, for now starting positions are assumed fixed and given.

Let $x_L(t)$ be position of the left leaf at time $t$. We have
\begin{equation}
  x_L(t) = x_L(0) + \int_{\tau=0}^t v_L(\tau) d\tau.
\end{equation}
\noindent (ditto for right leaf). Let $T_L(x)$ be the inverse of this function
(if we assume/enforce that velocity is always $>0$ then this inverse exists). Thus
$T_L(x)$ is the time that the left leaf crosses the position $x$. The fluence delivered to position
$x$, which we call $g(x)$ is the integration of the dose rate over the interval for which the position $x$ is not covered by either leaves, see Figure
\todo{Include figure}

\begin{equation}
  g(x) = \int_{t = T_R(x)}^{T_L(x)} d(t) dt
\end{equation}
\noindent where d(t) is the dose rate at time $t$.

The optimization problem is to find the velocities and the dose rate over time to get $g$ as close to $f$ as possible:

\begin{equation}
\mathrm{min} \int_{x=-1}^1 \left ( g(x) - f(x) \right )^2 dx
\label{opt}
\end{equation}


So, for example, the left leaf velocity is given by
$$
v_L(t) = \sum_{i=0}^3 c^L_i T_i(t).
$$
\noindent and we can do the same for the right leaf velocity and the dose rate as a function of time.
The optimization problem them becomes a search for the coefficients $c^L_i$, $c^R_i$, and $c^d_i$ that minimize the objective in Equation \ref{opt}.
